Y\-A\-M\-L in general

Y\-A\-M\-L(acronym of “\-Y\-A\-M\-L Ain't Markup Language”) is a human-\/readable data serialization language. Native Y\-A\-M\-L encodes data types that are specifically suitable for configuration. In this application, an enhanced format called C\-R\-O\-W-\/\-Y\-A\-M\-L is developed along with a series of advanced functionalities and data types to support the configuration of the workflows of N\-O\-A\-A.

The C\-R\-O\-W package has two-\/way conversion functions between Y\-A\-M\-L text files and configuration suite in memory. The configuration suite is essentially a python dictionary with all configuration settings. The settings are then parsed into config files, rocoto X\-M\-L and ec\-Flow suite.

Processing of Y\-A\-M\-L files

As a serialization language, Y\-A\-M\-L is extremely suitable to handle configuration practice because of it’s keyword-\/free syntax, highly human-\/readable nature, and the conveniency of handling multiple types of value within a single object.

This data structure is then read by a Python Y\-A\-M\-L toolbox called Py\-Y\-A\-M\-L. The primary function of Py\-Y\-A\-M\-L is two-\/way conversion between a Y\-A\-M\-L text file and a python object. Please note that Y\-A\-M\-L can be read by all major programming languages (Perl, Ruby, C++, Java…… ) when proper libraries are provided. Python is chosen for this project because it’s growing popularity in N\-C\-E\-P.

C\-R\-O\-W Y\-A\-M\-L is developed on top of Py\-Y\-A\-M\-L with the addition of several customized data types. User-\/defined data type is named alpha-\/numerically with “!” at the beginning when calling. C\-R\-O\-W defines several of these to handle calculations, conditionals, templates, times, dependencies, and some others.

Features\-: Only once. Though not recommended, C\-R\-O\-W enables users to specify a certain variable and override it at a later stage. But, only one value of a certain variable could be passed into the targeted workflow that is generated. All or none. C\-R\-O\-W always reads all Y\-A\-M\-L files at one time. This design helps ensure consistency within the Configuration Suite and Workflow Suite. Diagnosable. In C\-R\-O\-W, the experiment directory has effectively become a “checkpoint” for the configuration. This helps minimize human error and increase accountability, comparing with manually editing throughout experiment directory.

This file will be read by crow.\-config and become a eval\-\_\-dict object of python. There are two sections in the eval\-\_\-dict created, original contents are stored in \-\_\-child and calculated values are stored in \-\_\-cache.

C\-R\-O\-W based workflow suite

C\-R\-O\-W is a python based toolbox driven by a series of shell based utility scripts. In C\-R\-O\-W-\/enabled modeling systems, top level directory /workflow is the designated place to start working with workflow generation. Under this directory locates the sub-\/repository C\-R\-O\-W. Users need to perform a “git submodule” command to pull C\-R\-O\-W from the cloud. Outside C\-R\-O\-W repository, a user.\-yaml file and a case/ directory is needed to accommodate user and case settings. In addition to these, C\-R\-O\-W needs a set of background files to fulfill its functionality. These files are stored in workflow/defaults; workflow/workflow; workflow/runtime; workflow/platforms and /schema.

As stated before, C\-R\-O\-W uses a two-\/step approach to generate a workflow. The first step is called Configuration. In this step, C\-R\-O\-W will detect the running platform and read in all Y\-A\-M\-L files to make sure all required variables are properly set. If no problem occurs, a Configuration Suite will be generated. The Configuration Suite is a virtual object in the form of python ecal\-\_\-dict, which is essentially a database which only exists in memory. In the end, C\-R\-O\-W will rewrite all input Y\-A\-M\-Ls into experiment directory, which is created in the beginning of this step, and parse all configuration variables to config files into the experiment directory.

Namely, a configuration suite contains all configuration information of a given experiment, including default settings (clock, alarm) and most importantly, a series of tasks with dependencies between each other. A suite can be defined over one or more cycles, while each task has a section to define which cycle it will run.

C\-R\-O\-W Step 1, Configuration Flag-\/shaped boxes are text files; Circular bins are python objects; Yellow\-: User input files; Red\-: Background configuration files Green\-: Output files;

In the second step, named Generation, C\-R\-O\-W will read in all Y\-A\-M\-L files and generate the Configuration Suite again, while also populate all tasks, along with suite default settings, onto designated cycles to generate a Workflow Suite, with choice of Rocoto or ec\-Flow as workflow manager.

C\-R\-O\-W Step 2, Generation Flag-\/shaped boxes are text files; Circular bins are python objects; Yellow\-: User input files; Red\-: Background configuration files Green\-: Output files;

The basic component of a C\-R\-O\-W configuration suite is called a task. In a workflow, a task is a single element or step of the modeling system providing a certain functionality by a set of scripts (For example\-: gdasefcs01). Inside C\-R\-O\-W toolbox, a task is represented by python objects. Most tasks are associated with a J-\/\-Job script which handles submission of a supercomputing job card, with specific resource requests (time, cpu, memory ...). Some tasks come as an array called a “taskarray”, where multiple tasks performing the same functionality are grouped together (ie, gfs\-\_\-post\-\_\-\-X\-X, enkf\-\_\-forecast\-\_\-\-X\-X …..). Tasks will be instantiated as “jobs” when the workflow suite is generated and submitted to the workflow manager (For example\-: gdasefcs01 at 00 cycle and 06 cycle are two distinct jobs, but are derivatives of the same task gdasefcs01).

Fig 3. Detailed explanation of how tasks are defined will be given later

Example of task object creation\-:

This is what has been defined in task\-\_\-schema (rocoto section) \-: task\-\_\-schema\-: \&task\-\_\-schema !\-Template Rocoto\-: \# Variable name description\-: $>$-\/ \# Short description X\-M\-L to insert in the task definition, excluding the task tag itself, and the dependencies. type\-: string \# Variable type stages\-: \mbox{[} execution \mbox{]} \# Validation stage rocoto\-\_\-command\-: description\-: $>$-\/ Command to execute for this task when run in rocoto. This is inserted into the rocoto command tag for the task. type\-: string stages\-: \mbox{[} execution \mbox{]} …(\-More variables)... This is what has been inherited and defined in task\-\_\-template, only the rocoto parts are shown\-: task\-\_\-template\-: \&task\-\_\-template Template\-: $\ast$task\-\_\-schema \# Load task\-\_\-schema Rocoto\-: !expand $|$ \# “$|$” for multi-\/line string \section*{actual string with \{\} to be filled by other variables}

$<$command$>$sh -\/c '\{rocoto\-\_\-command\}'$<$/command$>$ \{partition.\-scheduler.\-rocoto\-\_\-accounting( partition\-\_\-specification,default\-\_\-accounting,accounting, jobname=task\-\_\-path\-\_\-var, outerr=rocoto\-\_\-log\-\_\-path, partition=partition.\-specification)\} rocoto\-\_\-command\-: !expand $>$-\/ \# actual string with \{\} \{rocoto\-\_\-load\-\_\-modules\} ; \{rocoto\-\_\-config\-\_\-source\} ; \{J\-\_\-\-J\-O\-B\-\_\-\-P\-A\-T\-H\}/\{J\-\_\-\-J\-O\-B\} $<$envar$>$$<$name$>$C\-D\-A\-T\-E$<$/name$>$

$<$cyclestr$>$$<$/cyclestr$>$

$<$/envar$>$ …(\-More lines of rocoto contents like the line above)... …(\-More variables)...

This is what has been inherited and defined in forecast\-\_\-task\-\_\-template, forecast\-\_\-task\-\_\-template\-: \&exclusive\-\_\-task\-\_\-template \# This is a task template $<$$<$\-: $\ast$task\-\_\-template \# It loads this template partition\-: !calc doc.\-accounting.\-exclusive\-\_\-partition \# Running on this partition default\-\_\-accounting\-: !calc partition.\-exclusive\-\_\-accounting\-\_\-ref \section*{default accounting}

J\-\_\-\-J\-O\-B\-: !expand '\{task\-\_\-path\-\_\-list\mbox{[}-\/1\mbox{]}.upper()\}' \# interface for J\-\_\-\-Job task\-\_\-type\-: forecast \# sticker

This is an actual task in the Y\-A\-M\-L configuration suite\-: jgdas\-\_\-forecast\-\_\-high\-: !\-Task \# This is a task $<$$<$\-: $\ast$forecast\-\_\-task\-\_\-template \# It loads this template Trigger\-: !\-Depend ( up.\-analysis.\-jgdas\-\_\-analysis\-\_\-high ) $|$ $\sim$ suite.\-has\-\_\-cycle('-\/6\-:00\-:00') \# This is the dependecies resources\-: !calc partition.\-resources.\-run\-\_\-gdasfcst \section*{This is resource request}

J\-\_\-\-J\-O\-B\-: J\-G\-L\-O\-B\-A\-L\-\_\-\-F\-O\-R\-E\-C\-A\-S\-T \# This is the associated J-\/\-Job

This is the task in Rocoto X\-M\-L generated from C\-R\-O\-W\-:

$<$task name=\char`\"{}gdas.\-forecast.\-jgdas\-\_\-forecast\-\_\-high\char`\"{} maxtries=\char`\"{}5\char`\"{}$>$ $<$command$>$sh -\/c ' source \$\-H\-O\-M\-Egfs/ush/load\-\_\-fv3gfs\-\_\-modules.sh exclusive ; module list ; source \$\-E\-X\-P\-D\-I\-R/config.base ; \$\-H\-O\-M\-Egfs/jobs/\-J\-G\-L\-O\-B\-A\-L\-\_\-\-F\-O\-R\-E\-C\-A\-S\-T'$<$/command$>$ $<$queue$>$$<$/queue$>$ $<$account$>$$<$/account$>$ $<$jobname$>$gdas.\-forecast.\-jgdas\-\_\-forecast\-\_\-high$<$/jobname$>$ $<$join$>$$<$cyclestr$>$//gdas.forecast.\-jgdas\-\_\-forecast\-\_\-high.\-log$<$/cyclestr$>$$<$/join$>$

$<$walltime$>$0\-:30\-:00$<$/walltime$>$ $<$memory$>$1024\-M$<$/memory$>$ $<$nodes$>$1\-:ppn=4+2\-:ppn=3$<$/nodes$>$

$<$envar$>$$<$name$>$C\-D\-A\-T\-E$<$/name$>$

$<$cyclestr$>$$<$/cyclestr$>$

$<$/envar$>$ $<$envar$>$$<$name$>$P\-D\-Y$<$/name$>$

$<$cyclestr$>$$<$/cyclestr$>$

$<$/envar$>$ $<$envar$>$$<$name$>$cyc$<$/name$>$

$<$cyclestr$>$$<$/cyclestr$>$

$<$/envar$>$ $<$envar$>$$<$name$>$E\-X\-P\-D\-I\-R$<$/name$>$

$<$/envar$>$ $<$envar$>$$<$name$>$D\-U\-M\-P$<$/name$>$

gdas

$<$/envar$>$ $<$envar$>$$<$name$>$R\-U\-N\-\_\-\-E\-N\-V\-I\-R$<$/name$>$

emc

$<$/envar$>$ $<$envar$>$$<$name$>$H\-O\-M\-Egfs$<$/name$>$

$<$/envar$>$ $<$envar$>$$<$name$>$H\-O\-M\-Eobsproc\-\_\-network$<$/name$>$

$<$/envar$>$ $<$envar$>$$<$name$>$H\-O\-M\-Eobsproc\-\_\-global$<$/name$>$

$<$/envar$>$ $<$envar$>$$<$name$>$H\-O\-M\-Eobsproc\-\_\-prep$<$/name$>$

$<$/envar$>$ $<$envar$>$$<$name$>$job$<$/name$>$

jgdas\-\_\-forecast\-\_\-high\-\_\-$<$cyclestr$>$$<$/cyclestr$>$

$<$/envar$>$

\begin{DoxyVerb}    <dependency>
          <or>
                <taskdep task="gdas.analysis.jgdas_analysis_high"/>
                <not>
                      <cycleexistdep cycle_offset="-06:00:00"/>
                </not>
          </or>
    </dependency>
\end{DoxyVerb}
 $<$/task$>$

Fig 4. Relationship between Configuration Suite and Workflow Suite

At this moment, configuration settings of each individual task are still coming from the corresponding config.\-xx files which is then linked to task. The purpose of this design is to make the transition easier for users of legacy configuration system.

User Input Files Format

The Y\-A\-M\-L files that require user inputs are all formatted in a classical and intuitive “\-Name\-: Value” convention. Python-\/style indentation rule is applied to differentiate multiple levels of variable sections. Top level named is “case” for case file; while “user\-\_\-places” and “accounting” is used for user file. Lower levels of contents include required and optional subsections. Order does not matter.

case\-: fv3\-\_\-settings\-: C\-A\-S\-E\-: C768 L\-E\-V\-S\-: 65

fv3\-\_\-enkf\-\_\-settings\-: C\-A\-S\-E\-: C384

……

User file

user.\-yaml

This file contains user-\/specific information of a given computing platform. Two sections “user\-\_\-places” and “accounting” are included.

A template of this file named “user.\-yaml.\-default” is included in the repository under workflow/. Users need to make their own “user.\-yaml” by modifying the values within the template when running C\-R\-O\-W for the first time.

user\-\_\-places\-: This section is of the same structure as “places” in case.\-yaml. The intention is to include settings that are more connected to the user other than the experiment. Such variables are marked red in the list. However, the user has the freedom to put all these variables into either one. When a certain variable get specified in both files, the one in this section will overwrite the one within case.\-yaml.

accounting\-: This section is designed to accommodate settings of supercomputer account and queue information.

Case file

\mbox{[}case name\mbox{]}.yaml

This file serves as the central place to configure the experiment parameters, located under /workflow/cases. Configurable variables are categorized by “group” for better handling and indexing. Complete list of variables for each group are given in the Appendix.

case name is a required argument of setup\-\_\-case.\-sh. When executed, the program will look for case name is workflow/cases directory to match the correct case file. A user may provide the full name of the yaml file as well (e.\-g. case\-\_\-name.\-yaml), the system can handle both.

“\-Default value\-: None” here means that no “group level” default values provided. But most of the variables still have an “individual” default value defined in the schema. Detail about the “default system” will be discussed later in the /default and /schema section.

fv3\-\_\-settings\-: Define spatial/vertical resolution and various physics parameters for deterministic and ensemble forecast jobs. Template\-: fv3\-\_\-settings\-\_\-template Default values\-: None

fv3\-\_\-gfs\-\_\-settings\-: settings for gfs (long forecast) model Template\-: fv3\-\_\-settings\-\_\-template Default values\-: None

fv3\-\_\-enkf\-\_\-settings\-: settings for D\-A ensemble forecast Template\-: fv3\-\_\-settings\-\_\-template Default values\-: fv3\-\_\-enkf\-\_\-defaults

fv3\-\_\-gdas\-\_\-settings\-: settings for D\-A deterministic forecast Template\-: fv3\-\_\-settings\-\_\-template Default values\-: None

For all four sections above, same variable list are provided. The gfs, enkf and gdas settings will inherit “fv3\-\_\-settings” if not specified separately.

schedvar\-: Scheduler-\/related variables. Template\-: schedvar\-\_\-schema Default\-: schedvar\-\_\-defaults

gfs\-\_\-output\-\_\-settings\-: Model output settings Template\-: gfs\-\_\-output\-\_\-settings\-\_\-template Default\-: gfs\-\_\-output\-\_\-settings\-\_\-defaults

data\-\_\-assimilation\-: Data assimilation configuration Template\-: data\-\_\-assimilation\-\_\-template Default\-: None

post\-: Post-\/processing configuration Template\-: post\-\_\-schema Default\-: None

downstream\-: Switches for turning on/off downstream jobs Template\-: downstream\-\_\-schema Default\-: downstream\-\_\-defaults

places\-: Settings of paths. Also need to specify which workflow files is used to create the final workflow. Template\-: places\-\_\-schema Default\-: default\-\_\-places

nsst\-: Near Sea Surface Temperature scheme settings. exclusive\-\_\-resources, shared\-\_\-resources, service\-\_\-resources\-: Resource specifications, default settings in platform file. Template\-: nsst\-\_\-schema Default\-: None

settings\-: general settings, like S\-D\-A\-T\-E, E\-D\-A\-T\-E and if “four cycle mode” will be used. Template\-: settings\-\_\-schema Default\-: default\-\_\-settings

archiving\-: Settings for archiving data Template\-: archive\-\_\-settings\-\_\-template Default\-: None

Suite\-\_\-overrides\-: Overriding values for the entire suite. No template is given.

Utility scripts\-:

Three most important utility scripts for the C\-R\-O\-W system, setup\-\_\-case.\-sh, make\-\_\-rocoto\-\_\-xml\-\_\-for.\-sh and make\-\_\-ecflow\-\_\-file\-\_\-for.\-sh, are already discussed in the section above. There are several additional scripts come with the package for various functionalities.

eclipse\-\_\-main.\-py

Usage\-: python3.\-6 eclipse\-\_\-main.\-py

This script is a python-\/based version of the setup\-\_\-case.\-sh. The purpose of this version is to provide the conveniency of launching I\-D\-E projects (ie\-: Eclipse) since most of them only support projects with only one language inside. It will do the following things\-: Validate platform; Set up C\-O\-M\-R\-O\-T directory under ptmp location, which is typically specified in user.\-yaml; Set up experiment directory under P\-R\-O\-J\-E\-C\-T\-\_\-\-D\-I\-R location, which is typically specified inside user.\-yaml; Create a configuration suite by reading all Y\-A\-M\-L files; A configuration suite is a python dictionary containing all configuration information. This suite is for validation and config file writing purpose and will only exist in memory. Write out all Y\-A\-M\-L files and config files into experiment directory.

ecflow\-\_\-main.\-py / rocoto\-\_\-main.\-py

Usage\-: python3.\-6 eclipse\-\_\-main.\-py python3.\-6 rocoto\-\_\-main.\-py

This file is the python equivalent to “make\-\_\-ecflow\-\_\-file\-\_\-for.\-sh” / “make\-\_\-rocoto\-\_\-file\-\_\-for.\-sh” as the entry point of “generation” step for ecflow or rocoto workflow generation.

worktools.\-py

This script is the central place to put top-\/level python functions regarding the handling of configuration definition documents and setting up the workflow.

Background Files Config

This directory stores the templates of all config files. For most modeling systems, it is usually a direct copy of “parm/config” directory.

Config files are bash shell scripts named “config.\-X\-X\-X” while “\-X\-X\-X” usually stands for a specific workflow task (example\-: config.\-fcst; config.\-post). These files store configurable variables for job “\-X\-X\-X” for later use in J-\/\-Jobs and ex-\/scripts. In addition to that, there are config.\-base and config.\-resource for general settings and resource allotment. When the model runs, each J-\/\-Job script will read in one or several of these files.

platform

This directory stores designated Y\-A\-M\-L files for each supported computing platform. For a given platform that supports the target workflow, the following parameters are defined\-:

Evaluate\-: this must be \char`\"{}false\char`\"{} to ensure disk space availability logic is not run unless this file is for the current platform. name\-: for machine name. detect\-: The detect criteria (like top level file structure) public\-\_\-release\-\_\-ics\-: location of input conditions that have been prepared for the public release C\-H\-G\-R\-P\-\_\-\-R\-S\-T\-P\-R\-O\-D\-\_\-\-C\-O\-M\-M\-A\-N\-D\-: rstprod data command. N\-W\-P\-R\-O\-D\-: location of N\-C\-E\-P operational directory. D\-M\-P\-D\-I\-R\-: location of the global dump data R\-T\-M\-F\-I\-X\-: location of the C\-R\-T\-M fixed data files used by the G\-S\-I data assimilation B\-A\-S\-E\-\_\-\-G\-I\-T\-: git command path. partitions\-: A comprehensive subsection, some machine comes with partitions like rdhpcs-\/jet. Resource allocation by partition are specified here. (partition detail) Least\-\_\-used\-\_\-temp\-: return the least used tmp directory.

workflow

This directory stores different type of workflow layouts associated with a certain modeling system. For example, cycled vs forecast-\/only for Global Workflow. In each workflow file, a sequence of tasks, or family of tasks, are given in an orderly manner. Job dependencies are also addressed in this file.

schema

This directory serves as a template for the configurations system for a given workflow. For each of the \mbox{[}case section\mbox{]}.yaml files within this directory, each file gives all possible values for a certain section of the workflow.

The schema/task.\-yaml sets up task interfaces and defines a template for tasks at the top level. After this step, workflow/runtime/task.\-yaml will implement all these interfaces (ie, “ecf\-\_\-file”, “rocoto\-\_\-command”... ) with actual content regarding workflow manager information, which is then instantiated /subclassed in the actual workflow. This process can be viewed as a practice of class inheritance in object-\/oriented programming.

Fig 3. Detailed explanation of how tasks are defined will be given later

runtime

This is the section for a series of task templates, as together with a series of suite default sections of rocoto and ec\-Flow. For example, templates for the rocoto “header” section

suite.\-yaml\-: Defines virtual object “suite\-\_\-default” which represents utilities and common sections which are shared through the entire workflow.

defaults

This directory stores default value for each part of the workflow. Comparing to the default values defined in /schema, this directory focuses on “group” default grouped by categories. These default values, if defined, will overwrite default values in schema. This section is organized in “group”s which is also housing all customized variables in the case file. Adding this section make configuration more manageable.

Fig 4. Overriding sequence, values of config variables

Files\-:

defaults/case.\-yaml

This file has the top-\/level logic to merge other Y\-A\-M\-L data structures into the document-\/level settings. It merges the contents of the case files, default files, platform file, and everywhere else, and applies any validation from the schema/ directory. This file should also be the first file to begin with when trying to build C\-R\-O\-W for a new modeling system.

defaults/downstream.\-yaml

This is a list of Y\-E\-S/\-N\-O switch to decide whether a downstream job should be included or not.

defaults/fv3\-\_\-enkf.\-yaml

This is the default settings of fv3 model and enkf mechanics.

defaults/gfs\-\_\-output\-\_\-settings.\-yaml

This file contains settings specific to model output of G\-F\-S.

defaults/places.\-yaml

This file is designated place to put paths. For example the home directory of G\-F\-S model and home directory of C\-R\-O\-W.

defaults/resources.\-yaml

This file stores default resource settings (number of cpu, length of time...) for each of the tasks in the workflow.

defaults/settings.\-yaml

This file is for general settings of global-\/workflow.

Testing

Unit test and regression test are provided within the C\-R\-O\-W package to help streamlining future development of the package. Unit tests focus on individual functions, while regression test proves if the output could be reproduced with updated code. Like the C\-R\-O\-W package itself, python 3.\-6 is a requirement to run these tests ( and to perform develop works for C\-R\-O\-W package )

Unit test\-: Enter the unit test directory\-: cd tests/unittests/slurm/ Execution command\-: sh run\-\_\-tests.\-sh

Sample output\-: test\-\_\-\-Aprun\-Cray\-M\-P\-I\-\_\-big (test\-\_\-\-Srun\-M\-P\-I.\-Test\-Aprun\-Cray\-M\-P\-I) ... I\-N\-F\-O\-:root\-:assertions not set yet ok test\-\_\-\-Aprun\-Cray\-M\-P\-I\-\_\-max\-\_\-ppn (test\-\_\-\-Srun\-M\-P\-I.\-Test\-Aprun\-Cray\-M\-P\-I) ... I\-N\-F\-O\-:root\-:assertions not set yet ok test\-\_\-\-First\-Max (test\-\_\-example\-Config.\-Test\-Example\-Config) ... ok test\-\_\-\-First\-True (test\-\_\-example\-Config.\-Test\-Example\-Config) ... ok …...

Unit test is done to ensure that all functions are in place and performs as designed. This test should be conducted after porting to a new platform, or working with a new version of the corresponding workflow (ie, global-\/workflow).

Regression test\-: Enter the regression test directory\-: cd tests/regtest/ Execution command\-: python3 regtest.\-py

Sample output\-: diff /scratch4/\-N\-C\-E\-P\-D\-E\-V/global/noscrub/\-Jian.Kuang/global-\/workflow/workflow/\-C\-R\-O\-W/tests/test\-\_\-data/regtest/cache/expdir/regtest\-\_\-tmp/../../../control/defs/regtest\-\_\-tmp /scratch4/\-N\-C\-E\-P\-D\-E\-V/global/noscrub/\-Jian.Kuang/global-\/workflow/workflow/\-C\-R\-O\-W/tests/test\-\_\-data/regtest/cache/expdir/regtest\-\_\-tmp/../../defs/regtest\-\_\-tmp Differing files \-: \mbox{[}'regtest\-\_\-tmp\-\_\-2016021000.\-def'\mbox{]}

diff /scratch4/\-N\-C\-E\-P\-D\-E\-V/global/noscrub/\-Jian.Kuang/global-\/workflow/workflow/\-C\-R\-O\-W/tests/test\-\_\-data/regtest/cache/expdir/regtest\-\_\-tmp/../../../control/expdir/regtest\-\_\-tmp /scratch4/\-N\-C\-E\-P\-D\-E\-V/global/noscrub/\-Jian.Kuang/global-\/workflow/workflow/\-C\-R\-O\-W/tests/test\-\_\-data/regtest/cache/expdir/regtest\-\_\-tmp/../../expdir/regtest\-\_\-tmp Identical files \-: \mbox{[}'\-\_\-main.\-yaml', 'case.\-yaml', 'config.\-anal', 'config.\-fcst', 'config.\-post', 'config.\-prep', 'names.\-yaml', 'platform.\-yaml', 'user.\-yaml', 'workflow.\-yaml'\mbox{]} Differing files \-: \mbox{[}'config.\-base', 'static\-\_\-locations.\-yaml', 'workflow.\-crontab', 'workflow.\-xml'\mbox{]} Common subdirectories \-: \mbox{[}'config', 'defaults', 'runtime', 'schema'\mbox{]}

C\-R\-O\-W regression test is essentially a repeatability test for workflow generation process. This test should be done before committing any changes to the C\-R\-O\-W master branch. 